% Paperformat
\documentclass[a4paper, 12pt]{scrartcl}
\usepackage{cite}
\usepackage{ragged2e}
\usepackage{url}
\usepackage{bm} % bold math
\usepackage{nccmath} % medium fraction with mfrac 
\usepackage{amsmath}
\usepackage{graphicx}
\usepackage{amsthm} % theorems
\usepackage{hyperref}

\newtheorem{theorem}{Theorem}[section]
\begin{document}

\begin{titlepage}
	\vspace*{\stretch{1}}
	\begin{center}
		{\Large\bfseries Bachelor Thesis}           \\[6.5ex]
		
		{\huge\bfseries Visualizing Dynamic Programming on Tree Decompositions}                  \\[6.5ex]
		
		\vspace{6ex}
				
		\textsc{\Large Martin Röbke}    \\[3ex]
		{\Large matriculation number: 3949819}    \\[2ex]
		{\Large martin.roebke@tu-dresden.de}    \\[2ex]
		\textsc{\large 
			}             \\[12ex]
		\vfill
		{\Large Technische Universität Dresden}               \\
		Faculty of Computer Science \\
		International Center For Computational Logic 		\\[5ex]
		
		{\Large Supervisor: Dr. Johannes Fichte}
		
		\vfill
		\today
	\end{center}
	\vspace{\stretch{2}}
\end{titlepage}



\section*{Abstract}
\vspace{4ex}
The present Bachelor thesis is about a practical and lightweight implementation of visualizing dynamic programming on tree decompositions.
I created the python-package tdvisu for the purpose of visualizing, teaching and analyzing the solving process of MSOL-problems using dynamic programming.
As reference implementations of dynamic programming on tree decompositions the projects \href{https://github.com/daajoe/GPUSAT}{GPUSAT} and \href{https://github.com/hmarkus/dp_on_dbs}{dpdb} were chosen.

\newpage

%  table
\tableofcontents

% chapter on next page
\newpage


\section{Introduction}

\newpage.
\section{Theory}
\newpage.
\section{My Visualization Project}
\newpage.
\section{Implementation in GPUSAT}
\newpage.
\section{Implementation in dpdb}
\newpage.
\section{Summary}

\bibliography{bibtex}{}
\bibliographystyle{ieeetr}

\end{document}