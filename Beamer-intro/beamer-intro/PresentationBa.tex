% !TeX spellcheck = en_US
\documentclass[c,8pt,xcolor...,x11names]{beamer}
\usepackage{icclslides}
\usepackage[latin1]{inputenc}
\usepackage[british]{babel}
\usepackage{amssymb}
\usepackage{latexsym}
\usepackage{rotate}
\usepackage{tikz}
\usepackage{verbatim}
\usepackage{colortbl}
\usepackage{booktabs}
\usepackage{ulem}
% \usepackage{arydshln}
%\usepackage{pdfpages}hord
\usepackage{graphicx} 
\usepackage{tikzsymbols}
\usepackage{tikz}
\usepackage{subcaption}
\usepackage[export]{adjustbox}
\usepackage{pdfpages}
\usetikzlibrary{positioning}

\tikzstyle{ele} = [circle, text centered, minimum width=1em, minimum height=3ex]

%% Uncomment to activate navigation symbols in the lower right of the pages:
\setbeamertemplate{navigation symbols}{}
%\setbeamercovered{transparent}

\renewcommand{\Myauthor}{Martin R\"obke}
\renewcommand{\Mytitle}{Visualizing Dynamic Programming On Tree Decompositions}
\usepackage{showexpl} 

\lstloadlanguages{[LaTeX]Tex} 
\lstset{% 
     basicstyle=\ttfamily\small, 
     commentstyle=\itshape\ttfamily\small, 
     showspaces=false, 
     showstringspaces=false, 
     breaklines=true, 
     breakautoindent=true, 
     captionpos=t 
} 

\begin{document}
\begin{frame}
%	Frequently asked questions during the defense of the bachelor thesis:
%What motivated you to write your Bachelor's thesis on exactly this topic?
%They often use the term X. Can you explain it in more detail?
%We miss the point Y in your work. What can you say to this?
%For what reason did you decide on the methodology you use?
%What added value has your research delivered?
%To what extent is your research relevant?
	\customtitle
	\begin{list2}
		\item {\sc What} was the motivation
		\item {\sc What} could be used otherwise?
		\item {\sc Who} benefits from visualization?
		\item {\sc Methodology} challenges and solutions
		\item {\sc What} could be developed next?
	\end{list2}
\end{frame}

%%%%%%%%%%%%%%%%%
\section{The task} 


%\begin{frame}
%	\frametitle{Task definition}
%\textit{Investigate how to automatically visualize dynamic programming algorithms based on existing implementations. Integrate your tool into at least one existing implementation, explain details on your implementation, how the visualization works, and show how this can be used for debugging algorithms.}
%
%%The algorithms exploit structural properties in the given problem instance and solve the problem faster if for example the treewidth of a graph representation is small, since they usually run exponentially in the treewidth and polynomially in the size of the input instance. In fact, recent research showed that implementations of dynamic programming algorithms can also compete with modern solvers and even outperform them in projected model counting.
%%Unfortunately, those algorithms are fairly hard to implement. While recent approaches have also investigated on allowing for easier implementations of dynamic programming algorithms on tree decompositions, implementations are still incredibly error prone, in particular, since they often involve bit fiddling and low level operations to make them run efficiently. Investigate how to automatically visualize dynamic programming algorithms based on existing implementations. Integrate your tool into at least one existing implementation, explain details on your implementation, how the visualization works, and show how this can be used for debugging algorithms.
%\medskip
%
%
%\end{frame}

%%%%%%%%%%%%%%%%%%%%%%
\section{Motivation}

\begin{frame}
	\frametitle{Motivation}
	
	\begin{itemize}
		\item DP-on-TD-algorithms can solve Model Counting and various combinatorial problems
		\item Implementations of those are competing with modern solvers
		\medskip
		\item {\Large But:} those are fairly hard to implement efficiently
%		as they involve bit fiddling
		\medskip
		\item Practical debug output soon gets very large
		\item Finding the cause of the problem is a time consuming challenge
	\end{itemize}

%Recent research showed that implementations of dynamic programming algorithms can also compete with modern solvers. \\
%Unfortunately, those algorithms are fairly hard to implement  to make them run efficiently.
%
%=> Debug output wird schnell un�bersichtlich/gro�
%Ursprung von Problemen auffinden => (MinVC) Beispiel
\end{frame}



%%%%%%%%%%%%%%%%%%%%%%%
%\section{Who should use this}
%\begin{frame}
%	\frametitle{Creating Visualization for:}
%	\begin{minipage}{0.44\textwidth}
%		\emph{Improving}
%		\begin{itemize}
%			\item examples for students
%			\item debugging and improving interaction of complex data-structures
%			\item hotspots
%		\end{itemize}\medskip
%		
%	\end{minipage}
%	\begin{minipage}{0.55\textwidth}
%		\begin{figure}
%			\includegraphics[width=\linewidth]{images/combined8.png}
%		\end{figure}
%	\end{minipage}
%	
%\end{frame}

%%%%%%%%%%%%%%%%%%%%%
\section{Background}
\begin{frame}
	\frametitle{Background}
	
=> Kombinatorische Prob. (l�sbare \#P Klasse schwerer(teilweise deutlich komplexer => Referenz Projected Model Counting Markus (hybride Erw dpdb) )) - Algos
\end{frame}

\subsection[TD]{Tree decomposition}
\begin{frame}
	
	\frametitle{Tree Decompositions}
	
	A tree decomposition is a tree obtained from an arbitrary graph
	s.t.\
	\begin{enumerate}
		\item {Each vertex} must occur in some
		\emph{bag}
		\item For {each edge}, there is a bag
		containing both endpoints
		\item %\emph{Connectedness condition}:\\
		{\emph{Connected}}: Subgraph ``restricted'' to any vertex must be connected
	\end{enumerate}

	%

\end{frame}
%{
%	\setbeamercolor{background canvas}{bg=}
%	\includepdf[pages=162-174]{"images/Lecture_pcgp_Summer_2019.pdf"}
%}

%%
\subsection{Instances}
%\begin{frame}
%	\frametitle{(Weighted) Model-Counting}
%\end{frame}

\begin{frame}
	\frametitle{Graphs for Boolean Formulas}
	\smallskip
	\begin{itemize}
		\item {\color{blue} Example set of CNF-clauses:}\smallskip\\
		{\tiny $\{\text{c1}=\{\text{v1},\text{v3},\neg \text{v4}\},\text{c2}=\{\neg \text{v1},\text{v6}\},\text{c3}=\{\neg \text{v2},\neg \text{v3},\neg \text{v4}\},\text{c4}=\{\neg \text{v2},\text{v6}\},\text{c5}=\{\neg \text{v3},\neg \text{v4}\},\text{c6}=\{\neg \text{v3},\text{v5}\},\text{c7}=\{\neg \text{v5},\neg \text{v6}\},\text{c8}=\{\text{v5},\text{v7}\}\}
			$
		}
	\end{itemize}
	\begin{figure}
		\includegraphics[height=0.5\textheight]{images/DAGraphs.png}
		\caption{The primal (left), incidence (middle) and dual (right) graph}
	\end{figure}
	%\footnotesize{Equivalent CNF}
	%{\tiny $(\neg \text{v1}\lor \neg \text{v3})\land (\neg \text{v1}\lor \text{v6})\land (\text{v1}\lor \neg \text{v4})\land (\neg \text{v2}\lor \neg \text{v5})\land (\neg \text{v2}\lor \text{v6})\land (\neg \text{v3}\lor \neg \text{v4})\land (\neg \text{v3}\lor \text{v5})\land (\neg \text{v5}\lor \neg \text{v6})\land (\text{v5}\lor \text{v7})$}
	
\end{frame}

%%


\subsection{Example Vertex cover}
\begin{frame}
	\frametitle[Vertex Cover]{Example: Vertex-Cover problem}

\end{frame}
%%
%\subsection[Courcelle]{Courcelle's theorem}
%\begin{frame}
%	\frametitle{Courcelle's theorem}
%
%%	\begin{quotation}
%%		Every graph property definable in monadic second-order logic (MSO) is decidable in linear time on graphs of bounded treewidth. \\
%%		\hfill {\small Courcelle, Bruno (1990)}\footnote{Courcelle, Bruno "The monadic second-order logic of graphs. I. Recognizable sets of finite graphs",\\ Information and Computation, 85 (1990) no. 1: 12-75}
%%	\end{quotation}
%
%	\medskip
%	For all $k \in \mathbb{N}$ and MSO-sentences F is the decision problem for a given graph G, whether $G \models F$ is true, in time $2^{p(tw(G))} \cdot |G|$ with a polynom p decidable.
%	\medskip
%	\begin{itemize}
%
%		\item \emph{drawback:} still expensive ($2^{p(tw G)}$, $2^{2^{(\#Q)}}$, large constants) \smallskip 
%		\item usage:
%
%	\end{itemize}
%	\begin{figure}
%		\includegraphics[height=0.2\textheight]{images/UsageCourcelle.gv.png}
%		\caption{Implementation of the theorem}
%	\end{figure}
%\end{frame}


%%%%%%%%%%%%%%%%%%%%%%%
\section{Implementations}

\subsection{gpusat2}
\begin{frame}
	\frametitle{gpusat2 - Improving Upon Previous Ideas }
	%Architecture of our DP-based solver for parallel execution. Yellow colored
	%boxes indicate tasks that are required as initial step for the DP-run or to nally read the
	%model count from the computed results. The parts framed by a dashed box illustrate the
	%DP-part. Boxes colored in red indicate computations that run on the CPU. Boxes colored
	%in blue indicate computations that are executed on the GPU (with waiting CPU).
	\begin{figure}
		\includegraphics[height=0.3\textheight]{images/gpusat2DP.png}
	\end{figure}
	\begin{minipage}{0.49\textwidth}
		\begin{itemize}

			\item only primal graph { \small (simpler solving DP)}
			\item customized tree decompositions
			\item adapted memory-management
			\item improved precision handling

		\end{itemize}
	\end{minipage}
%	\begin{minipage}{0.49\textwidth}
%		%Runtime for the top 5 sequential and all parallel solvers over all the #Sat
%		%instances with pmc preprocessor. The x-axis refers to the number of instances and the
%		%y-axis depicts the runtime sorted in ascending order for each solver individually.
%		\begin{figure}
%			\includegraphics[width=\linewidth]{images/gpusat2Runtime.png}
%		\end{figure}
%	\end{minipage}

\end{frame}
%%
\subsection{dpdb}
\begin{frame}
	\frametitle{dpdb}
	{\color{blue}Using databases for intermediate results} \medskip\\
	
	\begin{enumerate}
		\item Create graph representation
		\item Decompose graph
		\item Solve sub-problems
		\item Combine rows
	\end{enumerate}

	Generator SQL Qs => Datenbank Templating in Python
	%The idea of dpdb is to use database
	%management systems (DBMS) for table manipulation, which makes it (1) easy
	%and elegant to perform rapid prototyping for problems, and (2) allows to leverage
	%from decades of database theory and database system tuning. It turned out that
	%all the cases that occur in dynamic programming can be handled quite elegantly
	%with plain SQL queries. Our system dpdb can be used for both decision and
	%counting problems, thereby also considering optimization. We see our system
	%particularly well-suited for counting problems, especially, since it was shown
	%that for model counting (#Sat) instances of practical relevance typically have
	%small treewidth [23]. In consequence, we carried out preliminary experiments
	%on publicly available instances for
	\begin{minipage}{0.1\textwidth}
		\hfill
	\end{minipage}
	\begin{minipage}{0.35\textwidth}
		\begin{itemize}
			\item SAT
			\item \#SAT
			\item Vertex cover
		\end{itemize}
	\end{minipage}\hfill
	\begin{minipage}{0.54\textwidth}

		\begin{figure}
			\centering\hfill
			\begin{subfigure}[b]{\textwidth}
				\includegraphics[width=\linewidth]{images/dpdbSSat.png}
				\caption{Problem \#SAT}

			\end{subfigure}\hfill\\
			\begin{subfigure}[b]{\textwidth}
				\includegraphics[width=0.8\linewidth]{images/dpdbOCol.png}
				\caption{Problem \#o-Col}

			\end{subfigure}\hfill\\
			\begin{subfigure}[b]{\textwidth}
				\includegraphics[width=0.8\linewidth]{images/dpdbMinVC.png}
				\caption{Problem MinVC}

			\end{subfigure}
		\end{figure}

	\end{minipage}
	\medskip \\
	github: \url{https://github.com/hmarkus/dp_on_dbs}
\end{frame}


%%%%%%%%%%%%%%%%%%%%%%%
\section{Challenge}
\begin{frame}
	\frametitle{Challenge1}
	\medskip
	
	Generisches Datenformat /Strings
	Visu
	
	Arbeits Space gro�
	
	=> Anwendungen 
	
	
\end{frame}

\begin{frame}
	\frametitle{Challenge2}
	\medskip
	
	=> Wie robust ist die Datenverarbeitung in der Visu
	=> Was Gedanken bei der Visu waren
	
\end{frame}

\begin{frame}
	\frametitle{Challenge3}
	\medskip
	
\end{frame}
%%%%%%%%%%%%%%%%%%%%%%%
\section{Other Visualization}
\subsection{Software}
\begin{frame}
	\begin{figure}
		
		\includegraphics[width=0.25\linewidth]{images/gephi.png}
		\caption*{� Gephi.org - a tool for data analysts and scientists keen to explore and understand graphs.\footnote{https://gephi.org/}}
		\label{fig:gephi}
	\end{figure}
	
	
	%	Gephi is a tool for data analysts and scientists keen to explore and understand graphs. Like Photoshop? but for graph data, the user interacts with the representation, manipulate the structures, shapes and colors to reveal hidden patterns.
	
	\begin{figure}
		\includegraphics[width=0.25\linewidth]{images/tulip.png} \qquad
		\includegraphics[width=0.15\linewidth]{images/tulip_sample.jpeg}
		\caption*{
			Tulip - Better Visualization Through Research. \footnote{https://tulip.labri.fr/TulipDrupal/}}
		\label{fig:tulip}
	\end{figure}
	
	
	%	Written in C++ the framework enables the development of algorithms, visual encodings, interaction techniques, data models, and domain-specific visualizations. One of the goal of Tulip is to facilitates the reuse of components and allows the developers to focus on programming their application. This development pipeline makes the framework efficient for research prototyping as well as the development of end-user applications.
	
	\begin{figure}
		
		\includegraphics[width=0.1\linewidth]{images/visjs_logo.png} \qquad  \qquad
		\includegraphics[width=0.1\linewidth]{images/sigmajs.png} \qquad
		\includegraphics[width=0.15\linewidth]{images/force-graph.png}
		\caption*{ \qquad \qquad \qquad\footnote{https://neo4j.com/developer/tools-graph-visualization/} Vis.js \qquad  \qquad \qquad Sigma.js   \qquad
			vasturiano/3d-force-graph \footnote{https://github.com/vasturiano/3d-force-graph}}
		\label{fig:jsengines}
	\end{figure}
	
	
	%Neovis.js Popoto.js Vis.js Sigma.js ...
	%
	%Commercially licensed:
	%https://www.kineviz.com/graphxr/ 
	%
	%Dynamic Data Modeling, Time Series, Discover correlations, trends, and clusters.
	%%GraphXR is a start-to-finish web-based visualization platform for interactive analytics. For technical users, it?s a highly flexible and extensible environment for conducting ad hoc analysis. For business users, it?s an intuitive tool for code-free investigation and insight.
	%
	%https://github.com/vasturiano/3d-force-graph
	%%3-dimensional representation of a force-directed iterative layout, using 3d-force-graph. This component uses ThreeJS/WebGL for rendering and either d3-force-3d or ngraph for the 3D physics engine.
	%
	%%With this open source library, there are a couple of different components for handling the physics behind three dimensions and for actually rendering the visualization. It uses an iterative approach for rendering in 3D and creates stunning, interactive visualizations. The tool includes features for customizing styles of nodes and relationships, as well as container layouts, rendering controls, configuring simulation, and user interaction. The data structure required is similar to previous tools we have seen, with collections for nodes and relationships. 3d-force-graph also offers functionality for visualizations to use with virtual reality.
	%
	%``ELVIZ: A query-based approach to model visualization" 
	%%about an approach to visualization, generic regarding both the source model, and the kind and content of the visualization.
	%%Marie-Christin Harre, Jan Jelschen, and Andreas Winter. ELVIZ: A querybased
	%%approach to model visualization". In: Lecture Notes in Informatics
	%%(LNI), Proceedings - Series of the Gesellschaft fur Informatik (GI) (Jan.
	%%2014), pp. 105--120.
	
	=> Related Work Schluss / Wiss Arbeiten -> Nicht speziell Angeschaut / Format aus Solvern extrahiert - kann trotzdem sehr generisch sein (dpdb speziell)
\end{frame}

\subsection{Handcrafted}
\begin{frame}
	\frametitle{Visualization}
	\framesubtitle{Manually for gpusat}
	\begin{figure}
		\centering
		\includegraphics[width=0.6\linewidth]{images/DualDA43.png}
		\caption{Handcrafted \#SAT example-run from Markus Zisser\footnote{"Solving \#SAT on the GPU with Dynamic Programming and OpenCL",\\ Diploma Markus Zisser 2018 Technische Universit\"at Wien, p.33}}
		\label{fig:dualda43}
	\end{figure}
	
	=> Dynamic programming Grafiken / Verlaufsschema Kurz erkl�ren -> Beweise dazu sind aufw�ndig und recht speziell / 
\end{frame}

%%%%%%%%%%%%%%%%%%%%%%%
\section{Outlook}
\begin{frame}
	\frametitle{Outlook}
	\medskip
	for relevant problems the static graph visualization will become to complicated.
	
	https://data-science-blog.com/blog/2015/07/20/3d-visualisierung-von-graphen/
	%Bei besonders umfangreichen und zugleich vielf�ltigen Graphen ist eine Visualisierung in drei bzw. vier Dimensionen (x-, y-, z-Dimensionen + Zeit t) nicht nur sch�ner anzusehen, sondern kann auch sehr dabei helfen, ein Verst�ndnis (z. B. �ber Graphen-Cluster) zu erhalten.

	=> Automatische Methoden werden h�ufig schwerer als gedacht. 
	
	F�r tiefere Debugging Tasks m�sste evtl auch der Ansatz erneuert werden
	
	=> Was w�ren weitere Fragestellungen was man ansehen m�chte
\end{frame}
%%%%%%%%%%%%%%%%%%%%%%%

\begin{frame}
	\medskip
	\frametitle{Benchmark}
	{\color{blue}Performance of all three programs on \#SAT instances:} \medskip\\
	\begin{figure}
		\includegraphics[width=0.8\linewidth]{images/dpdbRuntime.png}
	\end{figure}
\end{frame}

\begin{frame}
	\frametitle{Visualization}
	\framesubtitle{Manually for dpdb}
	\begin{figure}
		\centering
		\includegraphics[width=\linewidth]{images/dpdbVisuSat.png}
		\caption{Handcrafted \#SAT example-run from dpdb\footnote{"Exploiting Database Management Systems and Treewidth for Counting",\\ Fichte, Hecher, Thier, Woltran} }
		\label{fig:dpdbVisuSat}
	\end{figure}
	
	
\end{frame}

%%%%%%%%%%%%%%%%%%%%%%%
\section{BIBLIOGRAPHY}
\begin{frame}
	\frametitle{Bibliography}
	\medskip
	
\end{frame}



%%%%%%%%%%%%%%%%%%%%%%%%
\bgroup
\setbeamercolor{background canvas}{bg=black}
\begin{frame}[plain]{}
\end{frame}
\egroup

\end{document}